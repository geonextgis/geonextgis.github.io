%-------------------------
% Resume in Latex
% Author : Krishnagopal Halder
% License : MIT
%------------------------

\documentclass[letterpaper,11pt]{article}

\usepackage{latexsym}
\usepackage[empty]{fullpage}
\usepackage{titlesec}
\usepackage{marvosym}
\usepackage[usenames,dvipsnames]{color}
\usepackage{verbatim}
\usepackage{enumitem}
\usepackage[hidelinks]{hyperref}
\usepackage{fancyhdr}
\usepackage[english]{babel}
\usepackage{tabularx}
\usepackage{hyphenat}
\usepackage{fontawesome}
\usepackage{graphicx} % Add this line for including graphics
\input{glyphtounicode}

%---------- FONT OPTIONS ----------
% sans-serif
% \usepackage[sfdefault]{FiraSans}
% \usepackage[sfdefault]{roboto}
% \usepackage[sfdefault]{noto-sans}
% \usepackage[default]{sourcesanspro}

% serif
% \usepackage{CormorantGaramond}
% \usepackage{charter}

\pagestyle{fancy}
\fancyhf{} % clear all header and footer fields
\fancyfoot{}
\renewcommand{\headrulewidth}{0pt}
\renewcommand{\footrulewidth}{0pt}

% Adjust margins
\addtolength{\oddsidemargin}{-0.5in}
\addtolength{\evensidemargin}{-0.5in}
\addtolength{\textwidth}{1in}
\addtolength{\topmargin}{-.5in}
\addtolength{\textheight}{1.0in}

\urlstyle{same}

\raggedbottom
\raggedright
\setlength{\tabcolsep}{0in}

% Sections formatting
\titleformat{\section}{
  \vspace{-4pt}\scshape\raggedright\large
}{}{0em}{}[\color{black}\titlerule \vspace{-5pt}]

% Ensure that generated pdf is machine-readable/ATS parsable
\pdfgentounicode=1

%-------------------------
% Custom commands

\newcommand{\resumeItem}[1]{
  \item\small{
    {#1 \vspace{-2pt}}
  }
}


\newcommand{\resumeSubheading}[4]{
  \vspace{-2pt}\item
    \begin{tabular*}{0.97\textwidth}[t]{l@{\extracolsep{\fill}}r}
      \textbf{#1} & #2 \\
      \textit{\small#3} & \textit{\small #4} \\
    \end{tabular*}\vspace{-7pt}
}

% Add this command for inserting a photo
\newcommand{\resumePhoto}[1]{\includegraphics[width=4cm]{#1}\\}

\newcommand{\resumeSubSubheading}[2]{
    \vspace{-2pt}\item
    \begin{tabular*}{0.97\textwidth}{l@{\extracolsep{\fill}}r}
      \textit{\small#1} & \textit{\small #2} \\
    \end{tabular*}\vspace{-7pt}
}


\newcommand{\resumeEducationHeading}[6]{
  \vspace{-2pt}\item
    \begin{tabular*}{0.97\textwidth}[t]{l@{\extracolsep{\fill}}r}
      \textbf{#1} & #2 \\
      \textit{\small#3} & \textit{\small #4} \\
      \textit{\small#5} & \textit{\small #6} \\
    \end{tabular*}\vspace{-5pt}
}


\newcommand{\resumeProjectHeading}[2]{
    \vspace{-2pt}\item
    \begin{tabular*}{0.97\textwidth}{l@{\extracolsep{\fill}}r}
      \small#1 & #2 \\
    \end{tabular*}\vspace{-7pt}
}


\newcommand{\resumeOrganizationHeading}[4]{
  \vspace{-2pt}\item
    \begin{tabular*}{0.97\textwidth}[t]{l@{\extracolsep{\fill}}r}
      \textbf{#1} & \textit{\small #2} \\
      \textit{\small#3}
    \end{tabular*}\vspace{-7pt}
}

\newcommand{\resumeSubItem}[1]{\resumeItem{#1}\vspace{-4pt}}

\renewcommand\labelitemii{$\vcenter{\hbox{\tiny$\bullet$}}$}

\newcommand{\resumeSubHeadingListStart}{\begin{itemize}[leftmargin=0.15in, label={}]}
\newcommand{\resumeSubHeadingListEnd}{\end{itemize}}
\newcommand{\resumeItemListStart}{\begin{itemize}}
\newcommand{\resumeItemListEnd}{\end{itemize}\vspace{-5pt}}

%-------------------------------------------
%%%%%%  RESUME STARTS HERE  %%%%%%%%%%%%%%%%%%%%%%%%%%%%


\begin{document}

%---------- HEADING ----------

\begin{minipage}[c]{0.6\textwidth}
    \textbf{\Huge \scshape Krishnagopal Halder} \\ \vspace{3pt}
    \small
    
    \faMobile \hspace{.5pt} \href{tel:6295024159}{+49 155 10743904}
    $|$
    \faAt \hspace{.5pt} \href{mailto:Krishnagopal.Halder@zalf.de}{Krishnagopal.Halder@zalf.de}
    $|$
    \faLinkedinSquare \hspace{.5pt} \href{https://www.linkedin.com/in/krishnagopal-halder}{LinkedIn}
    $|$

    \faGithub \hspace{.5pt} \href{https://github.com/geonextgis}{GitHub}
    $|$
    \faGlobe \hspace{.5pt} \href{https://geonextgis.github.io/}{Portfolio}
    $|$
    \faMapMarker \hspace{.5pt} \href{https://maps.app.goo.gl/JpcrK2GBHM3HWJqTA}{Müncheberg, Brandenburg, Germany} \\

    \faGraduationCap \hspace{.5pt} \href{https://scholar.google.com/citations?user=b02pfFMAAAAJ&hl=en&authuser=2}{Google Scholar}
    $|$
    \faMedium \hspace{.5pt} \href{https://medium.com/@geonextgis}{Medium}
    
\end{minipage}
\hfill
\begin{minipage}[c]{0.3\textwidth}
    \raggedleft
    \fbox{\includegraphics[width=4cm]{professional_dp.png}} % Adjust the width as needed
\end{minipage}

%----------- EDUCATION -----------

\section{Education}
  \vspace{3pt}
  \resumeSubHeadingListStart

    \resumeSubheading
      {Indian Institute of Technology Roorkee
      }{Roorkee, Uttarakhand, India}
      {M.Tech. in Disaster Mitigation and Management; \textbf{Dropout}}{Jul 2024 \textbf{--} Dec 2024}

    \vspace{4pt}
      
    \resumeSubheading
      {Vidyasagar University
      }{Midnapore, West Bengal, India}
      {M.Sc. in Remote Sensing and GIS;   \textbf{CGPA: 9.04/10.00}}{Nov 2022 \textbf{--} Jul 2024}

    \vspace{4pt}

    \resumeSubheading
      {Bankura Christian College
      }{Bankura, West Bengal, India}
      {B.Sc. in Geography;   \textbf{CGPA: 9.33/10.00}}{Jun 2019 \textbf{--} Aug 2022}

    \vspace{4pt}

    \resumeSubheading
      {Bankura Zilla School
      }{Bankura, West Bengal, India}
      {HSC- 12th Grade;   \textbf{Percentage: 95.80\%}}{Jun 2017 \textbf{--} Mar 2019}

      \normalsize \textit{Subjects: [Bengali, English, Geography, Computer Application, \\ 
                     \hspace{45pt} Philosophy, Mathematics]} 

    \resumeSubheading
      {Bankura Christian Collegiate School
      }{Bankura, West Bengal, India}
      {SSC- 10th Grade;   \textbf{Percentage: 90.43\%}}{Apr 2016 \textbf{--} Feb 2017}
      
      \normalsize \textit{Subjects: [Bengali, English, Mathematics, Physical Science, \\ 
                     \hspace{45pt} Life Science, History, Geography]} 
    
  \resumeSubHeadingListEnd



%----------- SKILLS -----------

\section{Skills}
  \vspace{3pt}
  \resumeSubHeadingListStart
    \small{\item{
        
        \textbf{Languages:}{ Python, SQL, JavaScript, R} \\ \vspace{3pt}
        
        \textbf{Technologies:}{ Google Earth Engine, PyTorch, TensorFlow, Scikit-learn, Git, GitHub, PostGIS} \\ \vspace{3pt}
        
        \textbf{Methodologies:}{ Machine Learning, Deep Learning, Geospatial Analysis, Image Processing, Statistical Analysis} \\ \vspace{3pt}

        \textbf{Softwares:}{ ArcGIS Pro, QGIS, Photoshop, Adobe Illustrator} \\ \vspace{3pt}
        
    }}
  \resumeSubHeadingListEnd



%----------- EXPERIENCE -----------

\section{Experience}
  \vspace{3pt}
  \resumeSubHeadingListStart

    \resumeSubheading
      {Leibniz Centre for Agricultural Landscape Research (ZALF)}{Müncheberg, Brandenburg, Germany}
      {Research Scientist}{Jan 2025 \textbf{--} Present}
      
     \resumeItemListStart
        \resumeItem{Working within the Research Platform \textit{Simulation and Data Science}, under the Working Group \textit{Multi-Scale Modelling and Forecasting (MSM)}, contributing to the ``Leibniz-Lab Systemic Sustainability (LL SYSTAIN)" project.}
        \resumeItem{Developing machine learning workflows for multi-scale crop yield forecasting and scenario analysis integrating climate, soil, and biodiversity indicators.}
        \resumeItem{Applying geospatial data science techniques to process and harmonize large Earth observation and agricultural datasets (e.g., Sentinel, MODIS, soil grids).}
        \resumeItem{Collaborating with interdisciplinary teams to co-design sustainability indicators linking agricultural productivity and ecological resilience.}
    \resumeItemListEnd  
    
    \resumeSubheading
      {Institute of Crop Science and Resource Conservation (INRES)}{University of Bonn, Germany}
      {Remote Intern}{Feb 2023 \textbf{--} Feb 2024} %Full-time}
     \resumeItemListStart  
        \resumeItem{Developed and validated ML/DL-based models (e.g., XGBoost, CNNs) for regional and continental-scale crop yield prediction using satellite-derived vegetation indices and climate data.}
        \resumeItem{Performed phenological and climate feature extraction using Google Earth Engine (GEE), Python (NumPy, Pandas, Scikit-learn), and TensorFlow for time-series analysis.}
        \resumeItem{Processed multi-terabyte Earth observation datasets via cloud computing platforms to derive agricultural insights and optimize model input features.}
        \resumeItem{Authored comprehensive technical reports and visualized key results using Python (Matplotlib, Plotly), contributing to publications on data-driven agricultural sustainability.}
    \resumeItemListEnd  
    
  \resumeSubHeadingListEnd



%----------- AWARDS & ACHIEVEMENTS -----------
  \section{Awards \& Achievements}
  \vspace{3pt}
      \resumeItemListStart

          \resumeItem{\textbf{Secured All India Rank (AIR) 36 in Geomatics Engineering Paper} of GATE 2024 conducted by Indian Institute of Science, Bangalore. \textit{(Mar 2024)}} \vspace{-2pt}

          \resumeItem{\textbf{Ranked among the top 0.01 percentile students} statewide among 777,000 candidates in the Higher Secondary Certificate (HSC) examination with a test score of 479/500. \textit{(Jun 2019)}} \vspace{-2pt}

        \resumeItemListEnd


%----------- PROJECTS -----------

\section{Projects}
    \vspace{3pt}
        \resumeItemListStart
            \resumeItem{\textbf{LL SYSTAIN}: Leibniz-Lab Systemic Sustainability. Biodiversity, Climate, Agriculture and Food within Planetary Boundaries. \href{https://www.ll-systain.org/}{(Website)}
            \\\textit{(Funded by \textbf{Leibniz Association})} \textit{(Jan 2025 \textbf{--} Present)}}

            \resumeItem{\textbf{SynPAI}: Synergising Process-Based and Mchine Learning Models for Accurate and Explainable Crop Yield Prediction along with Environmental Impact Assessment. \href{https://synpaim.wordpress.com/news/}{(Website)}
            \\\textit{(Funded by \textbf{Biotechnology and Biological Sciences Research Council (BBSRC)})} \textit{(Feb 2024 \textbf{--} Present)}}

            \resumeItem{\textbf{AgML}: Machine learning for agricultural modelling. \href{https://www.agml.org/}{(Website)} \textit{(Jan 2024 \textbf{--} Present)}}

          \resumeItemListEnd



%----------- WORKSHOPS & SEMINARS -----------

\section{Workshops \& Seminars}
    \vspace{3pt}
        \resumeItemListStart
            \resumeItem{\textbf{Oral Presenter:} ``Mapping Land Surface Phenology for Sustainable Food Systems: A Multiscale, Open-Source Approach Using Multi Source Remote Sensing", at \textbf{Sustainable Agriculture for Food Security and Global Health}, GBPUA\&T, India, Oct 2025.} \vspace{-2pt}
            
            \resumeItem{\textbf{Workshop Moderator:} ``Hy4Cast – Hybrid modelling leveraging Artificial Intelligence for fine-scale crop yield forecasts", at \textbf{Tropentag 2025}, University of Bonn, Germany, Sep 2025.} \vspace{-2pt}
            
            \resumeItem{\textbf{Poster Presenter:} ``Advancing Crop Yield Predictions: The Potential of Diffusion Models in Machine Learning for Agriculture", at \textbf{EGU General Assembly}, Vienna, Apr-May 2025.} \vspace{-2pt}
            
            \resumeItem{\textbf{Poster Presenter:} ``Transforming Low-Resolution CORINE Data into High-Resolution Landscape Maps with Semi-Supervised Deep Learning", at \textbf{EGU General Assembly}, Vienna, Apr-May 2025.} \vspace{-2pt}
            
            \resumeItem{\textbf{Oral Presenter:} Indo-German Workshop on \textbf{Resilient Food Systems: AI, Remote Sensing, and Crop Models in Harmony (R-FARM)}, University of Bonn, Germany, Feb 2024}. \vspace{-2pt}

            \resumeItem{\textbf{Oral Presenter:} International Seminar on \textbf{Recent Advancement in Geographical Studies - A Multidimensional Outlook}, Department of Geography, Rampurhat College, Birbhum, WB, India, Sep 2023.} \vspace{-2pt}

            \resumeItem{\textbf{Workshop Attendee:} Training Course on \textbf{Aquifer Mapping and Management}, Central Ground Water Board, Salt Lake, Kolkata, Feb 2022.}

          \resumeItemListEnd




%----------- PUBLICATIONS -----------
\section{Publications}
\vspace{3pt}

\begin{enumerate}[{start=11,label=[\arabic*]\addtocounter{enumi}{-2}}]

    \item Alsafadi, K., Srivastava, A. K., \textbf{Halder, K.,} el al., (2025).A machine learning framework for modeling and upscaling mangrove carbon productivity (ML-MCP). \\
    \textit{(\textbf{Agricultural and Forest Meteorology; IF: 5.7; Q1})}
    \href{https://doi.org/10.1016/j.agrformet.2025.110821}{(Link)}\vspace{-2pt}

    \item Utthasini, M., Ilampooranan, I., Singh, S. K., Kanga, S., Kumar, P., \textbf{Halder, K.,} el al., (2025). Enhancing landslide susceptibility mapping in the Himalayas: geospatial and machine learning with explainable AI. \\
    \textit{(\textbf{Gondwana Research; IF: 8.6; Q1})}
    \href{https://doi.org/10.1016/j.gr.2025.08.003}{(Link)}\vspace{-2pt}
    
    \item \textbf{Halder, K.,} Srivastava, A. K., el al., (2025). A robust and scalable crop mapping framework using advanced machine learning and optical and SAR imageries. \\
    \textit{(\textbf{Smart Agricultural Technology; IF: 5.7; Q1})}
    \href{https://doi.org/10.1016/j.atech.2025.101354}{(Link)}\vspace{-2pt}
    
    \item Xing, L., Zhao, R., Sun, H., Tan, Z., Fang, Q., Li, M., \textbf{Halder, K.,} el al., (2025). Propagation patterns of different degree meteorological droughts across the Yangtze River Basin: a three-dimensional drought feature identification approach with Copula modeling. \\
    \textit{(\textbf{Journal of Hydrology; IF: 6.3; Q1})}
    \href{https://doi.org/10.1016/j.jhydrol.2025.133857}{(Link)}\vspace{-2pt}

    \item \textbf{Halder, K.,} Ewert, F., et al., (2025). High-Resolution Maize Yield Mapping across Africa using Earth Observation and Machine Learning, Deep Learning, and Foundation Model. \\
    \textit{(\textbf{Research Square preprint})}
    \href{https://doi.org/10.21203/rs.3.rs-6935279/v1}{(Link)}\vspace{-2pt}

    \item \textbf{Halder, K.,} Srivastava, A.K., et al., (2025): Improving landslide susceptibility prediction through ensemble Recursive Feature Elimination and meta-learning framework. \\
    \textit{(\textbf{Scientific Reports; IF: 3.8; Q1})}
    \href{https://www.nature.com/articles/s41598-025-87587-3}{(Link)}\vspace{-2pt}

    \item Shi, Y., Han, L., Zhang, X., Sobeih, T., Gaiser, T., Thuy, N. H., Behrend, D., Srivastava, A.K., \textbf{Halder, K.,} et al., (2025). Deep Learning Meets Process-Based Models: A Hybrid Approach to Agricultural Challenges. \\
    \textit{(\textbf{arXiv preprint})}
    \href{https://doi.org/10.48550/arXiv.2504.16141}{(Link)}\vspace{-2pt}
    
    \item Patra, S., Saha, A., Pal, S. C., Islam, A. R. M. T., \textbf{Halder, K.,} et al (2025). Highlighting the role of traditional paddy for sustainable agriculture and livelihood: issues, policy intervention and the pathways.\\
    \textit{(\textbf{Discover Sustainability; IF: 3.0; Q1})}
    \href{https://doi.org/10.1007/s43621-025-00989-1}{(Link)}\vspace{-2pt}
    
    \item Srivastava, A.K., Rahimi, J., Alsafadi, K., Vianna, M., Enders, A., Zheng, W., Demircan, A., Dieng, M.D.B., Salack, S., Faye, B., Singh, M., \textbf{Halder, K.,} et al., (2025):     Dynamic Modelling of Mixed Crop-Livestock Systems: A Case Study of Climate Change Impacts in sub-Saharan Africa. \\
    \textit{(\textbf{Scientific Reports; IF: 3.8; Q1})} \href{https://www.nature.com/articles/s41598-024-81986-8}{(Link)} \vspace{-2pt}

    \item \textbf{Halder, K.,} Ghosh, A., et al., (2024): SAR-driven flood inventory and multi-factor ensemble susceptibility modelling using Machine Learning Frameworks.\\
    \textit{(\textbf{Geomatics, Natural Hazards, and Risk; IF: 4.5; Q1})} \href{https://doi.org/10.1080/19475705.2024.2409202}{(Link)} \vspace{-2pt}
    
    \item \textbf{Halder, K.,} Srivastava, A.K., et al., (2024): Application of bagging and boosting ensemble machine learning techniques for groundwater potential mapping in a drought-prone agriculture region of eastern India.\\
    \textit{(\textbf{Environmental Sciences Europe; IF: 6.0; Q1})} \href{https://link.springer.com/article/10.1186/s12302-024-00981-y}{(Link)} \vspace{-2pt}
    

    % \item Alsafadi, K., Bashir, B., Srivastava, A.K., \textbf{Halder, K.,} et al., (2024): Machine Learning-Based Models Are Enhancing Prediction Of Mangrove Carbon Production (ML-MCP). \\ 
    % \textit{(Submitted in \textbf{Climate and Atmospheric Science; IF: 8.5; Q1})} \vspace{-2pt}

    % \item Utthasini, M., Ilampooranan, I., Singh, S.K., Kumar, K., Kanga, s., \textbf{Halder, K.,} et al., (2024): Advanced Geospatial and Machine Learning Techniques for Landslide Susceptibility Mapping in the Himalayas: Integrating Explainable AI for Improved Geohazard Assessment. \\ 
    % \textit{(Submitted in \textbf{Natural Hazards; IF: 3.3; Q1})} \vspace{-2pt}

    % \item Pal, S.C., Patra, S., Biswas, T., Srivastava, A.K., Chatterjee, U., \textbf{Halder, K.,} et al., (2024): Highlighting the role of traditional paddy for sustainable agriculture and livelihood: Issues policy intervention and the pathways. \\ 
    % \textit{(Submitted in \textbf{Discover Sustainability; IF: 2.3; Q2})} \vspace{-2pt}

\end{enumerate}


%----------- RELEVANT COURSEWORK -----------

% \section{Relevant Coursework}
%   \vspace{2pt}
%   \resumeSubHeadingListStart
%     \small{\item{
%         \textbf{Major coursework:}{ Materials Science, Electrical Circuits I-II, Digital System Design, Numerical Methods, Probability Theory, Electronics I-II, Signals and Systems, Electromagnetic Field Theory, Energy Conversion, System Dynamics and Control, Communication Engineering, Pattern Recognition, Introduction to Digital Signal Processing, Introduction to Digital Communications, Introduction to Database Systems, Introduction to Image Processing, Machine Vision} \\ \vspace{3pt}
        
%         \textbf{Minor coursework:}{ Discrete Computational Structures, Introduction to Object-Oriented Programming, Data Structures and Algorithms, Computer Organization, Fundamentals of Software Engineering}
%     }}
%   \resumeSubHeadingListEnd



%----------- CERTIFICATES -----------

% \section{Certificates}
  % \resumeSubHeadingListStart
    
    % \resumeOrganizationHeading
      % {Procter \& Gamble VIA Certificate Program}{Feb 2022}{Project Management and Personal Productivity}
    
  % \resumeSubHeadingListEnd



%----------- ORGANIZATIONS -----------

% \section{Organizations}
  % \resumeSubHeadingListStart
    
    % \resumeOrganizationHeading
      % {Institute of Electrical and Electronics Engineers (IEEE)}{Feb 2022 -- Present}{Student Member}
    
  % \resumeSubHeadingListEnd



% %----------- HOBBIES -----------

% \section{Hobbies}
%   \vspace{3pt}
%   \resumeSubHeadingListStart
%     \small{\item{Painting, Reading, Cricket, Football, Horology}}
%   \resumeSubHeadingListEnd



%----------- REFEREES -----------

\section{Referees}
  \vspace{3pt}
  \resumeSubHeadingListStart
    % \item \textbf{Amit Kumar Srivastava} (Senior Scientist) \\ Institute of Crop Science and Resource Conservation, University of Bonn \\
    % Katzenburgweg 5, 53115 Bonn, Germany. \\
    % Email: amit@uni-bonn.de \\
    % Phone: +49-0228-73-2881 (Office)

    \item \textbf{Dr. Frank Ewert} (Prof. Dr. agr.) \\ 
    Leibniz Centre for Agricultural Landscape Research (ZALF)\\
    Eberswalder Str. 84, 15374 Müncheberg, Germany\\
    Email: Frank.Ewert@zalf.de \\
    Phone: +49 33432 82200 (Office)
    
    \item \textbf{Dr. Amit Kumar Srivastava} (Senior Scientist) \\ 
    Leibniz Centre for Agricultural Landscape Research (ZALF)\\
    Eberswalder Str. 84, 15374 Müncheberg, Germany\\
    Email: AmitKumar.Srivastava@zalf.de \\
    Phone: +49 33432 82173 (Office)

    % \item \textbf{Dr. Subrata Pan} (Associate Professor) \\ Department of Geography, Bankura Christian College \\
    % Bankura, West Bengal, India. \\
    % Email: pansubrata1@gmail.com \\
    % Phone: +91 94342 02363 (Personal)

    % \item \textbf{Subodh Chandra Pal} (Assistant Professor) \\ 
    % Department of Geography, University of Burdwan\\
    % Purba Bardhaman, West Bengal, India. \\
    % Email: subodhrsgis@gmail.com \\
    % Phone: +91 98321 10304 (Personal)
            
  \resumeSubHeadingListEnd



%-------------------------------------------
\end{document}